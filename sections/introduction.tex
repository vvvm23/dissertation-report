% TODO: distill high level trends rather than honing in on particular papers.
% What makes specific methods work with characteristics / limitations? Where is
% the research gap?

% TODO: research questions are a bit fragmented -- lacking a united and salient
% narrative.
An ideal deep generative model would satisfy three key requirements:
high-quality samples, sample diversity via mode coverage, and computational
inexpensive sampling. Arguably, there are other desirable properties such as a
meaningful latent space and exact likelihood calculation, however no current
generative model can satisfy all three requirements -- let alone additional
attractive properties -- thus forming the so-called generative modelling
trilemma~\cite{xiao2021trilemma}.

Models such as \glspl{gan}~\cite{goodfellow2014gan}
excel at high-quality and fast sampling, but often fail to model the entire data
distribution due to not directly optimising for likelihood -- using an
adversarial loss as a proxy. \Glspl{vae}~\cite{kingma2013vae} offer
excellent mode coverage and fast sampling speeds, but the resulting samples are
often blurry even at small resolutions, and have little hope of scaling to
greater resolutions like \glspl{gan}.

\Gls{ar} models such as PixelSnail~\cite{chen2017snail}, Image
Transformer~\cite{parmar2018image}, and DALL-E~\cite{parmar2018image} have
demonstrated respectable sample quality and mode coverage, even including
zero-shot image generation~\cite{ramesh2021dalle}. However, they are
computationally expensive to sample from, requiring many network iterations,
making them infeasible for interactive applications. \Glspl{ddpm}~\cite{ho2020ddpm}
and \glspl{sbm}~\cite{song2019sbm,song2020sde,song2021mlt} produce
samples that rival or even exceed the quality of \glspl{gan}~\cite{dhariwal2021ddpm}
whilst still providing good mode coverage, but are still plagued by potentially
requiring thousands of network evaluations.

Vector-quantized image
modelling~\cite{oord2017vqvae,razavi2019generating,esser2021taming} alleviates
sampling speed issues in \gls{ar} methods by reducing the spatial dimension at
which \gls{ar} sampling occurs. This results in excellent quality samples whilst
improving sampling speeds, but often requires a two-stage approach and still
does not match the speed of \glspl{gan}. Recent work has applied \gls{vq}
methods to diffusion models~\cite{bondtaylor2021unleashing} allowing for fast
parallel decoding. Other work does not use vector-quantized spaces, but latent
spaces nonetheless, to accelerate
sampling~\cite{xiao2021trilemma,vahdat2021sbmlatent}.

From this brief overview of generative modelling literature, it is clear that
indeed no single model satisfies all three conditions. This therefore motivates
research into explicitly addressing this trilemma. In this work, we move towards
such a solution, beginning from existing work applying generative models to
discrete latents. This provides an excellent starting point in terms of sample
quality and a respectable amount of mode coverage, but an unacceptably slow
sampling speed despite the reduced spatial dimension of the discrete latent
space. We directly address this issue by instead sampling discrete latents using
modern \gls{nar} generative models in an effort to close the gap, sampling speed
wise, with constant iteration complexity models such as \glspl{gan}.
Specifically, we use discrete \gls{sundae}~\cite{savinov2022stepunrolled} to
gradually denoise samples from a uniform prior into one that matches the prior
over the discrete latent space defined by a pre-trained \gls{vqgan} model. This
forms our first research question, whether \gls{sundae} models is suitable
paradigm for sampling discrete latent codes. As a result of our studies, we find
that indeed they are highly suited for this task.

These discrete \gls{sundae} models have only previously be applied to language
modelling tasks~\cite{savinov2022stepunrolled}. In their work, they used
transformer~\cite{vaswani2017attention} architectures to form their autoencoder.
In parallel, a noticeably more efficient variant of transformers -- the
Hourglass Transformer~\cite{nawrot2021hierarchical} -- was released, which has a
hierarchical architecture aimed at language and image modelling tasks, though
potentially flawed on the latter. This raises a second research question,
whether the approach introduced in~\citet{savinov2022stepunrolled} is also
suited for latent modelling, and whether the hierarchical approach introduced
in~\citet{nawrot2021hierarchical} can be further improved upon. As a result of
our studies, we find that although they are suited for latent modelling, there
are modifications that can improve performance further. These include
modifications to the down- and up-sampling operations, as well as axial
positional embeddings in hourglass architectures~\cite{nawrot2021hierarchical}
when operating on multi-dimensional sequences; removal of causal masking, and
differences in parameters for \gls{sundae} training and sampling when applied to
discrete latents rather than natural language.

Given a fast sampling and an efficient transformer architecture, we are left
with a highly scaleable generative model, with respect to number of layers and
spatial resolution of the input. Specifically, only a minority of the layer are
operating at the same resolution as the input, reducing the cost of expensive
self-attention across the input. Conversely, we can scale the number of layers
more cheaply by adding layers only at the downsampled resolution, allowing for
considerably larger models with only a minor computational cost. This
raises a third and final research question, whether our new approach allows for
the generation of extremely high resolution images in a time frame that would
permit fully- or near-interactive use. Again, as a result of our studies, we
find that this is indeed the case, proven by the synthesis of $1024 \times 1024$
images in as few as 2 seconds on a consumer-grade GPU. To our knowledge, this is
the fastest sampling model at this resolution, with the exception of pure
\gls{gan}-based approaches. This required the training of our own \gls{vqgan}
model, which was, also to our knowledge, the largest \gls{vqgan} model trained
in terms of input size.

%In this work, we aim to move towards satisfying all three key requirements for
%an ideal generative model. Like previous work, we use a vector-quantized image
%model to reduce the spatial dimension of the signal we wish to sample. We then
%apply new language models to instead model the distribution of discrete latents,
%ultimately obtaining a powerful prior over the discrete latents. With this and a
%discrete latent decoder, we obtain the final generative model, allowing sampling
%of images in a very low number of steps. We go further, and train our own
%VQ-GAN~\cite{esser2021taming} at resolutions higher than ever trained before to
%our knowledge, ultimately allowing for the sampling of $1024 \times 1024$ RGB
%images in only two seconds on a consumer-grade GPU.

Our project objectives were as follows:
\begin{itemize}
    \item Apply \gls{sundae} to the task of \acrshort{vq} latent generation and
        ultimately image generation, and perform analysis into their advantages
        of being used for this task over existing approaches, including both
        \gls{ar} and \gls{nar} solutions.

    \item Analyse the methods used in \citet{savinov2022stepunrolled} and
        \citet{nawrot2021hierarchical} and design modifications that make these
        approaches more suitable for \acrshort{vq} latent modelling.

    \item Attempt to scale our model to extremely high resolution images and
        analyse successes and failures when operating at such resolutions. 

\end{itemize}

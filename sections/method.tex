% TODO: figures for the following:
% - Latent dataset generation
% - High level overview of hourglass
% - 2d aware modifications: 2d rotary embeddings and downsampling in two
%   directions (showing failings of previous)
% - show training procedure (original z, corruption function, unrolled
%   training). show samples at each step? or save for eval
% - show sampling procedure (random z_0, unrolled steps, show samples, binary
%   mask)
% - show inpainting procedure (modification to binary mask)
% - show some nice samples!

\subsection{Latent Dataset Generation}
We use the standard two-stage scheme for vector-quantized image
modelling~\cite{oord2018neural,razavi2019generating,esser2021taming,bondtaylor2021unleashing} using
VQ-GAN~\cite{esser2021taming} as our feature extractor. Where such models are
available, we use pretrained VQ-GANs for our experiments. For higher resolution
experiments (for example, FFHQ-1024~\cite{karras2019stylebased}), pretrained
models are not available and so training our own VQ-GAN was necessary (see
\S\ref{sec:megagan}).

The second stage is to learn a discrete prior model over these latent variables.
To enable this, we must first build a latent dataset using our trained VQ-GAN.
Formally, given a dataset of images $\imageDataset$, a VQ-GAN encoder
$\vqganEncoder$ with downsample factor $\vqganDownsample$, and
vector-quantization codebook $\vqganCodebook$ with number of codewords
$\vqganNbLatents$, trained on $\imageDataset$, we define our latent dataset
$\latentDataset$ as:
\begin{equation}
    \latentDataset = \{\vqganCodebook(\vqganEncoder(\image)) \mid \image \in \imageDataset \}
\end{equation}
where $\image \in \real{3 \times H \times W}$ is a single element of the image
dataset and $\latent = \vqganCodebook(\vqganEncoder(\image)) \in \{1, \dots,
v\}^{h \times w}$ is the corresponding discrete latent
representation. In other words, each $\vqganDownsample \times \vqganDownsample$
pixels in $\image$ is mapped to a single discrete value from $1$ to
$|\vqganCodebook|$ (which in turn, corresponds to a vector $\codebookVector \in
\vqganCodebook$),
resulting in a latent representation of shape $\frac{H}{f} \times \frac{W}{f} =
h \times w$.

We then use $\latentDataset$ to train a discrete prior over the latents. Coupled
with the VQ-GAN decoder $\vqganDecoder$, we obtain a powerful generative model. 

\subsection{2D-Aware Hourglass Transformer}
Inspired by successes in hierarchical transformers for generative language
modelling~\cite{nawrot2021hierarchical}, we modify their architecture for use
with discrete latent representations of image data. We will later use this
architecture as the discrete prior over the VQ-GAN latents. 

Hourglass transformers have been seen to efficiently handle long-sequences,
outperform existing models using the same computational budget, and meet the
same performance as existing models more efficiently by using an explicit
hierarchical structure~\cite{nawrot2021hierarchical}. The same benefits should
also apply to vector-quantized image modelling.

%Our modifications are 2D-aware downsampling, axial rotary embeddings, and
%removal of causal modelling constraints.

%\subsubsection{2D-Aware Downsampling}

% TODO: add a figure demonstrating this
\textbf{2D-Aware Downsampling} -- The original formulation of hourglass
transformers~\cite{nawrot2021hierarchical} introduced both upsampling and
downsampling layers, allowing the use of hierarchical transformers in tasks that
have output sequence length equal to the input sequence length. However,
applying their proposed resampling strategies directly on the vector-quantized
image may not be the best strategy. Resampling is applied to flattened token
sequence, meaning that the corresponding two-dimensional vector-quantized image
is actually resampled more in one axis compared to the other. In their work they
did not address this, except for experiments on
ImageNet32~\cite{russakovsky2015imagenet} where they resampled with a rate of
$\hourglassRate=3$, corresponding to three colour channels.

In our formulation, we instead reshape the flattened sequence back into a
two-dimensional form and then apply resampling equally in the last two axes.
With a resampling rate of $\hourglassRate$ we apply $\sqrt{\hourglassRate}$ in each axis. We found this to
significantly improve the performance of the discrete prior model, and suspect a
similar approach could improve performance if applied to pixels directly, which
we leave for future work.

%\subsubsection{Axial Rotary Embeddings}

\textbf{Rotary Positional Embeddings}~\cite{su2021roformer} are a good default
choice for injecting positional information into transformer models, requiring
no additional parameters. Additionally, they can be easily extended to the
multi-dimensional case~\cite{rope-eleutherai} which we do here. Though
transformers are clearly capable of learning that elements far apart in a
flattened sequence may be close in a multi-dimensional final output, we find
that explicitly extending positional embeddings to the multi-dimensional case to
provide a modest boost in performance.

%\subsubsection{Removal of Causal Constraints}

\textbf{Removal of Causal Constraints} -- In the original autoregressive
formulation of hourglass transformers, great care was taken to avoid information
leaking during resampling, and hence making the model
non-causal~\cite{nawrot2021hierarchical}. We use a non-autoregressive method
which is therefore not causal. Hence, in our approach we do not make any special
considerations to avoid information leaking into the future.

\subsection{Non-Autoregressive Generator Training}
% TODO: perhaps should move this to the prior work part?

We follow the same process for training the discrete prior model step-unrolled
denoising autoencoders (SUNDAE)~\cite{savinov2022stepunrolled}. Given a uniform
prior $p_0$ over our latent space $\latentSpace = \{1, \dots,
\vqganNbLatents\}^N$ where $N=h\cdot w$, consider the Markov process
$\latent_t \sim \sundae(\cdot \vert \latent_{t-1})$ where $\sundae$ is a neural
network
parameterised by $\sundaeParameters$, then $\{\latent_t\}_t$ forms a Markov
chain. This gives a $t$-step transition function:
\begin{equation}\label{eq:markov}
    p_t(\latent_t \vert \latent_0) = \sum\limits_{\latent_1, \dots,
    \latent_{t-1} \in \latentSpace} \prod\limits^t_{s=1} \sundae(\latent_s | \latent_{s-1})
\end{equation}\cite{savinov2022stepunrolled}
and, given a constant number of steps $\markovSteps$, our model distribution
$p_\markovSteps(\latent_\markovSteps \vert \latent_0)p_0(\latent_0)$ -- which is
clearly intractable.

Instead, they propose an \textit{unrolled denoising} training method that uses a
far lower $\markovSteps$ than is used for
sampling~\cite{savinov2022stepunrolled}. To compensate, they unroll the Markov
chain to start from corrupted data produced by a \textit{corruption
distribution} $\latent' \sim \corruptionDistribution(\cdot \vert \latent)$ rather than from the prior $p_0$ so the model
encounters samples more akin to those seen during the full unroll at sample
time~\cite{savinov2022stepunrolled}. The original work defaults to $\markovSteps
= 2$ during training, noting also that a single step would be similar to the
training strategy of BERT~\cite{devlin2019bert} but would lead to worse
performance as seen in earlier work using BERT as a random field language
model~\cite{wang2019bert}. A good default of $\markovSteps = 2$ held true in our
experiments on codebook latents.

The training objective of SUNDAE is simply the average of all reconstruction
losses $\lossFunction{1:T} = \frac{1}{T} \left(\lossFunction{1} + \dots +
\lossFunction{T} \right)$ of the chain after $t$ steps, which is shown to form
an upper bound on the actual negative
log-likelihood~\cite{savinov2022stepunrolled}. Taking more steps $\markovSteps$
leads to a minor improvement in performance, but considerably slows down
training time~\cite{savinov2022stepunrolled} and memory usage.

We follow the original choice of corruption
distribution~\cite{savinov2022stepunrolled}: sample some
proportion $r \sim U[0, 1]$, randomly selecting positions according to this
proportion, and then sampling at these selected positions tokens random tokens
from $\{1, \dots, \vqganNbLatents\}$.

A key advantage of this approach on discrete latents over other
non-autoregressive methods is that we start from random tokens, rather than a
dedicated ``masking'' token~\cite{bondtaylor2021unleashing}. Using a masking
token means that $\markovSteps \leq h\cdot w$ as we must at minimum unmask a single
token. Additionally, the model is able to ``change its mind'' about generated
positions during the sampling process, allowing it to make fine-grained
adjustments later in the sampling process.

% TODO: a figure could be really nice too.

\subsection{Generating High-Resolution Images}
% TODO: discuss our various sampling strategies and any additional findings over
% original SUNDAE

% TODO: we found the best sampling parameters to be a little different to their
% one.
% TODO: also discuss annealing temperature, low temperature but fast, high
% temperature but slower, etc.
% in general, the sampling process for SUNDAE is highly tuneable, and probably
% dataset dependent..
% we also stop early if nothing has changed.

During sampling, we simply sample sequentially $\latent_t \sim \sundae(\latent_t
\vert \latent_{t-1})$ for a constant number of steps $\markovSteps$, beginning
randomly from $\latent_0$~\cite{savinov2022stepunrolled}. The original work
proposed a number of improved strategies for sampling in smaller number of
steps, including low-temperature sampling and updating a random subset of
tokens~\cite{savinov2022stepunrolled}, rather than all simultaneously.

Sampling, however, with a lower temperature can reduce the diversity of the
resulting samples. To alleviate this, we instead anneal the temperature down
from a high value ($\approx 1.0$) down to a lower value towards the end of the
sampling process. We found this retained the fast sampling speed whilst also
improving diversity.

In certain latent sampling configurations, updating only a random subset of
tokens does improve performance. However, we found that for low step scenarios
($\markovSteps<20$) that all tokens must be able to be updated in order to
produce meaningful samples before the maximum number of steps is reached. Hence
in these cases, we do not follow this strategy.

Additionally, if a individual sample does not change between step $t-1$ and $t$,
we freeze it, preventing any further change. If all samples are frozen, sampling
may terminate early, further improving sampling speed with little cost to
sample quality.

Once sampling has terminated, the sampled latent code $\latent_T$ can be given
to the VQGAN decoder $\vqganDecoder$ to produce a final sample $\sample$.

\subsection{Arbitrary Pattern Inpainting}
% TODO: discuss how to inpaint with a trained model
% TODO: add more citations in general for other NAR
% TODO: clear advantage of NAR models against AR, as we are not limited to
% causal inpainting
% TODO: initial description may be more suited for prior work.
As noted in the original work~\cite{savinov2022stepunrolled} and other
non-autoregressive solutions~\cite{bondtaylor2021unleashing} one clear advantage
of non-autoregressive models is that they are not limited to causal inpainting.
In general, they support arbitrary inpainting masks and can draw on context in
both the past and the future, enabling them to perform inpainting tasks not
possible with autoregressive sampling.

Given a sampled image $\sample \in \real{3 \times H \times W}$ we can mask a
proportion of it using a pixel-level binary mask $\pixelMask \in \{0, 1\}^{H
\times W}$. By taking $\vqganDownsample \times \vqganDownsample$ regions of
$\pixelMask$ and applying a logical and in them, we can obtain a latent level
mask $\vqMask \in \{0,1\}^{h \times w}$. We then sample as normal from the
latents, allowing the model full context, but only update regions that were
masked according to $\vqMask$. Like with sampling, we then use $\vqganDecoder$
to decode the sampled latent code, producing the output $\sample$.

\subsection{Training a megapixel VQ-GAN}
% TODO: details about FFHQ1024 VQGAN.
% this kind of stuff might be better in "evaluation" though

Training at higher resolutions usually means greater computational requirements
and sampling speeds. With an autoregressive model, the sampling speed can be
especially immense, even with an auxiliary vector-quantized image
model~\cite{esser2021taming}. With a non-autoregressive model however, one
question to explore is whether sampling at very high resolutions becomes
feasible.

To answer this question, we train a larger variant of VQGAN with
$\vqganNbLatents = 8192$ operating on $1024 \times 1024$ RGB images. To our
knowledge, this is the highest resolution VQGAN has been applied
to~\cite{esser2021taming}. Once trained, we can train SUNDAE samplers on the
latents as before, the only difference being an increased sequence length --
greater than was ever tested in the original
work~\cite{savinov2022stepunrolled}.

\label{sec:megagan}
